\chapter{Conclusion}
\label{sec:conclusion}

Contemporary Intel processors need to reduce their frequency when executing \gls{AVX} code in order to maintain stability. This has the effect that the performance of heterogeneous workloads with only some specific vectorizable code paths can suffer from these frequency reductions when using \gls{AVX} on these chips. In this work, we wanted to explore the possibility of optimizing Intel's frequency scaling behavior by designing a software-based reimplementation that is supposed to allow us to test alternate algorithms without needing to modify the hardware itself. To be able to do this, we first needed to perform an analysis of the reclocking implementation found in the hardware as there is only very little documentation by Intel available.

In said analysis we found that, for most demanding \gls{AVX} instructions, downclocking is triggered as soon as a single operation was executed and takes about 25 -- \SI{50}{\micro\second}, depending on the heaviness of the executed instructions. As soon as the last instruction has retired, a processor will wait for around \SI[quotient-mode=fraction]{2/3}{\milli\second} before reverting an \gls{AVX}-induced frequency reduction. We used a simplified model based on these results to construct our reimplementation \textsc{avxfreq}, which proved to reflect the model reasonably well. We also were able to show that exploiting an application's knowledge about what it is going to do next can improve performance in a heterogeneous program by \SI{15}{\percent} during scalar phases. However, we also experienced several drawbacks with our reimplementation approach. These include unpredictable delays when raising the frequency, the lack of ways to fully disable the reclocking done by the hardware itself and the impossibility to precisely measure time starting from the point when the last \gls{AVX} instruction in a consecutive streak was executed. Some of these issues may be alleviated or circumvented with further work. As of now, we can not provide a conclusive answer to the feasibility of our idea, as more exploration would be required through implementations of alternative reclocking governors.

\section{Future Work}
\label{sec:conclusion:futurework}

Although we were able to show that improved performance in heterogeneous workloads through better reclocking is theoretically possible, research on this topic is nowhere complete. In this section, we will present several ideas and open issues that may pose interesting topics for future work.

\subsubsection{Alternative Reclocking Algorithms}
\label{sec:conclusion:futurework:algorithms}

We designed our reimplementation \textsc{avxfreq} with the goal of being able to test alternative reclocking algorithms. So far we implemented a simple design which uses oracle-style foresight to switch frequencies by having user-space programs tell the kernel whenever a phase with \gls{AVX-512} heavy instructions starts or ends. This approach allowed us to estimate what performance improvement may be possible in theoretical best case. However, of course this is not feasible for real-world application as one of the main tasks of an operating system is to hide hardware implementation details such as the need for reclocking from software. We have outlined an approach where the operating system may profile durations of different execution phases in user-space threads to make predictions about the future length of \gls{AVX} and scalar phases that may be used to estimate whether immediately raising the processor's frequency after an \gls{AVX} phase would have a positive impact on overall performance.

\subsubsection{Analysis on Other Processors}
\label{sec:conclusion:futurework:otherprocessors}

Our analysis in \Cref{sec:analysis} was conducted using an Intel Core i9-7940X processor from the \textit{Skylake (Server)} generation. However, even though older chips from the \textit{Haswell} and \textit{Broadwell} generations only supported \gls{AVX2} and no \gls{AVX-512} \cite{intelhaswell} \cite{intelxeonscalabledeepdive}, they still reduced their frequency when executing demanding \gls{AVX2} instructions, albeit to a smaller extent. Given that we found several discrepancies in the reclocking behavior between what Intel claims in their manuals and what we saw in our analysis results, it may be plausible that Intel's implementation has changed between \textit{Haswell} and \textit{Skylake} and their documentation is simply outdated. Running our tests with a \textit{Haswell} processor could therefore yield interesting insights on why Intel has chosen to implement reclocking the way they did in \textit{Skylake}.

Further, not all \textit{Skylake (Server)} cores are equally equipped with the same amount of \gls{AVX-512} execution units. Our processor had two units per core, whereas other processors from this generation only have one \cite{intelxeonscalabledeepdive}. As more units incur larger power consumption and higher energy density, it is possible that chips which only have one unit per core may behave differently. For the sake of completeness in our analysis and because of potentially different requirements to alternative reclocking algorithms, we should run our framework again on such a processor.

Finally, towards the end of 2019, \textit{Icelake} processors will be the first in the consumer desktop and mobile markets to have \gls{AVX-512} \cite{thicelake}. As Intel makes this instruction set available to a broader range of customers and given that \textit{Icelake} will be the first large microarchitectural overhaul since \textit{Skylake} along with a leap in process technology, they possibly have made changes to their reclocking algorithm. Again, it may be interesting to test our analysis framework on an \textit{Icelake} chip to find potential differences that may have an impact on our and other approaches to optimizing the performance of heterogeneous workloads.

\subsubsection{SMT Support in Our Analysis Framework}
\label{sec:conclusion:futurework:smt}

% TODO better reference
As described in \Cref{sec:analysis:design:kernel}, our analysis framework does not support running on processors with \glsreset{SMT}\gls{SMT} enabled. Although we can not conceive of a reason why a processor should behave differently with respect to reclocking when \gls{SMT} is enabled, it is still possible that deviations exist. Given that in nearly all real-world scenarios \gls{SMT} would be enabled, we should implement \gls{SMT} support and verify that there are indeed no differences.

\subsubsection{Huge Pages in the Analysis User-Space Component}
\label{sec:conclusion:futurework:hugepages}

In \Cref{sec:analysis:design:userspace} we described how cache coherency effects impacted the results of our analysis framework when run on multiple cores with many pages, each \SI{4}{\kibi\byte} in size. We alleviated the impact by using less pages and instead having a jump at the end to make them loop. However, this theoretically introduces small inaccuracies to the results as the instruction stream becomes slightly heterogeneous. Modern \gls{x86} processors alternatively also support pages that are \SI{2}{\mebi\byte} large \cite{intelsdmsysprogguide}, which may be a viable alternative to improve the accuracy of our results.

\subsubsection{Instruction Mixtures}
\label{sec:conclusion:futurework:mixtures}

For our analysis, we ran each test using just one specific instruction, repeated many times. However, as Intel hints in their optimization manual \cite{inteloptimizationmanual}, the reclocking is actually triggered depending on the instruction mix executed over a window of several clock cycles. Using data science and machine learning techniques, it may be possible to create a model that can very precisely predict what mixtures will cause a frequency reduction.

\subsubsection{Compiler Optimizations Using Analysis Results}
\label{sec:conclusion:futurework:compiler}

Due to the dangers of possible performance degradation for heterogeneous workloads, most compilers do not generate \gls{AVX-512} code unless explicitly instructed to do so \cite{lemire2018avx512}. Even Intel's own C and C++ compiler that previously used to generate \gls{AVX-512} instructions aggressively whenever possible has become very conservative in this regard. With a deeper understanding about when and under what circumstances \gls{AVX}-induced reclocking happens, it may be possible for compilers to make use of \gls{AVX} more often without needing to worry about negative impacts. Thanks to the analysis we have done, much of the required knowledge is now available, while the rest may be achieved through further work as outlined in this chapter.

\subsubsection{Reverse Engineering of AVX Frequency Offsets}
\label{sec:conclusion:futurework:reversing}

In our discussion in \Cref{sec:evaluation:discussion} we have identified the impossibility to completely disable \gls{AVX} reclocking in hardware as a major drawback for our reimplementation approach. However, it is unknown how our motherboard's \gls{UEFI} configures \gls{AVX} frequency offsets in the processor. It may be possible to find out through some reverse engineering efforts though, and this could theoretically reveal a way to achieve full suppression of the processor's own reclocking mechanism.

\subsubsection{Hardware Limitations}
\label{sec:conclusion:futurework:hardware}

On Intel \textit{Skylake} \glspl{CPU} it is possible to measure executed floating-point vector instructions, however, no similar performance events exist for other vector instructions. Although this is not within our scope of control, we would like to see event types for all possible types of vector instructions in future processors. It might even make sense to have events that count executed instructions per instruction set. Being able to count all instruction types would be necessary for building a complete reimplementation that can be tested with real-world programs.

Another obstacle preventing a precise software reimplementation of Intel's reclocking algorithm is that we are not able to start measuring time immediately after the last \gls{AVX} instruction has retired. As outlined in \Cref{sec:evaluation:discussion}, a vast improvement here is likely possible with future \textit{Icelake} processors by using improved capabilities of the \gls{PEBS} facility. However, for our purposes it would be optimal if the \gls{PMU} would simply record the \gls{TSC} time-stamp on every assertion of a performance event.

Further, although this is not too much of a problem for us, we are unable to control the power gating of vector registers and vector execution units. In \Cref{sec:evaluation:discussion} we have discussed that the worst conceivable impact are slightly reduced frequencies due to exceeded power budgets. However, it may be interesting to measure the implications of alternative reclocking algorithms for power consumption, which is only inaccurately possible without being able to disable power supply to the gated units upon reverting a frequency reduction.