\chapter{Abstract}
\label{sec:abstract}

While using the \acrlong{AVX} (\acrshort{AVX}) on current Intel \gls{x86} processors allows for great performance improvements in programs that can be parallelized by using vectorization, many heterogeneous workloads that use both vector and scalar instructions expose degraded throughput when making use of \gls{AVX2} or \gls{AVX-512}. This effect is caused by processor frequency reductions that are required to maintain system stability while executing \acrshort{AVX} code. Due to the delays incurred by frequency switches, reduced clock speeds are attained for some additional time after the last demanding instruction has retired, causing code in scalar phases directly following \acrshort{AVX} phases to be executed at a slower rate than theoretically possible.

We present an analysis of the precise frequency switching behavior of an Intel \textit{Syklake (Server)} \gls{CPU} when \acrshort{AVX} instructions are used. Based on the obtained results, we propose \textsc{avxfreq}, a software reimplementation of the \acrshort{AVX} frequency selection mechanism. \textsc{avxfreq} is designed to enable us to devise and evaluate alternative algorithms that govern the processor's frequency smarter with regard to workloads that mix both \acrshort{AVX} and scalar instructions. Using an oracle mechanism built atop \textsc{avxfreq}, we show that the performance of scalar phases in heterogeneous workloads can theoretically be improved by up to \SI{15}{\percent} with more intelligent reclocking mechanisms.