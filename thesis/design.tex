\chapter{Design}
\label{sec:design}

The ultimate goal of this work is to find optimizations to the reclocking algorithm implemented in Intel \glspl{CPU} for \gls{AVX}-induced frequency reductions. In \Cref{sec:analysis} we have conducted a thorough analysis of the implementation in order to be able to conceive possible optimizations. Concretely, ideas for such algorithms may be derived from what is done in power management: we have described how the problems and possible gains of \gls{AVX} reclocking resemble the ones of power management in \Cref{sec:background:powermanagement}. The strategies employed there, such as profiling and the use of statistic models, and previous research on these approaches may be useful to develop a \gls{AVX} reclocking mechanism that can improve performance for heterogeneous workloads. Any ideas, however, would remain purely theoretical unless we are able to implement and experimentally test them. In this chapter, we present \textsc{avxfreq}, a modified version of the Linux kernel's \gls{DVFS} driver for Intel processors that tries to mimic Intel's hardware-based reclocking behavior. We can then use this \emph{reimplementation} to explore, implement and evaluate optimization approaches.

\section{Reimplementation}
\label{sec:design:reimplementation}

To be able to implement alternative \gls{AVX} algorithms, we want to reproduce Intel's \gls{AVX} reclocking mechanism in software. This means that we want to take over control over \gls{AVX}-induced frequency reductions from the hardware. To achieve this goal, we have built \textsc{avxfreq}, a reimplementation based on a model obtained through the results of the analysis we performed in \Cref{sec:analysis}. \textsc{avxfreq} counts retired \gls{AVX} instructions over time periods to govern the frequency, therefore, akin to our analysis framework, \textsc{avxfreq} makes use of the processor's \gls{PMU}, the features of which we outlined in \Cref{sec:analysis:design:pmu}. This section describes the pre-existing kernel driver our reimplementation is based on, the modifications we made to it and how we use the \gls{PMU} to reach our goals.

\subsection{The intel\_pstate Driver}
\label{sec:design:reimplementation:intel_pstate}

\textsc{Avxfreq} was implemented by modifying \gls{Linux}'s \texttt{intel\_pstate} module as it is found in version 5.1.0 of the kernel. This driver is part of the \textit{\texttt{cpufreq}} subsystem which contains all of the kernel's \gls{DVFS} drivers and provides generic policies that govern the frequency selection based on system load and the user's wishes \cite{cpufreq}. %and is responsible for managing \gls{HWP} as described in \Cref{background} and selecting \glspl{P-state} based on the system's load.

On modern Intel \glspl{CPU} with support for \glsreset{HWP}\gls{HWP} (as described in \Cref{sec:background:dvfs}), the \texttt{intel\_pstate} driver usually does not do much: it enables \gls{HWP} using the processor's \texttt{IA32\_PM\_ENABLE} \gls{MSR}, reads the \texttt{IA32\_HWP\_CAPABILITIES} \gls{MSR} to determine the available \glspl{P-state} on each core and then only manages the \texttt{IA32\_HWP\_REQUEST} \gls{MSR} to hint the hardware about changes of the user's preferences on, for example, minimum and maximum performance. \cite{intelsdmsysprogguide} \cite{intelpstate}. When \gls{HWP} is enabled, the operating system loses its ability to precisely manage the processor's frequency. However, using \gls{HWP} is not a strict requirement -- as long as \gls{HWP} is not explicitly activated by the computer's \gls{UEFI} or the operating system, traditional software-based performance scaling remains entirely possible. For users desiring to retain the pre-\gls{HWP} behavior, the \texttt{intel\_pstate=no\_hwp} flag exists. When this flag is passed to the \gls{Linux} kernel's command line during startup, \texttt{intel\_pstate} will not enable \gls{HWP}, and thus is still able to exercise complete control over the processor's speed. This is an important corner stone for \textsc{avxfreq}: we need to be able to fully govern \gls{P-state} selection in order to switch frequencies by ourselves in response to changing \gls{AVX} load conditions.

\subsection{AVXFreq}
\label{sec:design:reimplementation:avxfreq}

To obtain a minimal working prototype that allows us to evaluate the general feasibility of a software-based approach to \gls{AVX} reclocking optimization, we chose to implement a simplified version of Intel's algorithm that we derived from the \gls{CPU}'s behavior when executing \SI[number-unit-product=-]{512}{\bit} \texttt{vfmaddsub132pd} instructions, as obtained from the analysis in \Cref{sec:analysis}. For this instruction type, we found that a single instruction is sufficient to trigger a frequency reduction. After  approximately \SI{25}{\micro\second}, level~1 is reached, and -- in the case of the unrolled version -- about \SI{27}{\micro\second} later, the core runs at level~2. As soon as no heavy \gls{AVX} instructions have been executed anymore for \SI[quotient-mode=fraction]{2/3}{\milli\second}, the processor reverts the frequency reduction and returns to level~0. %For our implementation, we want to ignore the delays during downclocking and simply switch to level 1 as soon as a single instruction was executed

For reasons of practical convenience, \textsc{avxfreq} was implemented to be opt-in: the \texttt{intel\_pstate=avxfreq} flag must be passed on \gls{Linux}'s command line, otherwise \texttt{intel\_pstate} behaves just like it normally would in its unmodified version. Note that this flag also implies \texttt{intel\_pstate=no\_hwp}, i.e., it also disables \gls{HWP} for the reasons explained above.

To manage license levels, we added a per-core state variable that contains the virtual license level and adapted the method that communicates the selected \gls{P-state} to the processor to offset the \gls{P-state} according to the frequency reduction that would be induced by the hardware in this license level. This way, the driver's frequency selection mechanism stays unmodified and fulfills its tasks unaware of and unaffected by our \gls{AVX} reclocking. To ensure that changes to the virtual license level immediately take effect, we invoke a recalculation every time, and thus an update of the target \gls{P-state}.

During the boot process, \gls{Linux}'s \texttt{perf} subsystem configures performance interrupts to be delivered as \glspl{NMI} \cite{kernelx86eventscore}. In our analysis framework, we wanted to minimize overhead caused by interrupt handling to keep our workload as free from non-\gls{AVX} code as possible, and therefore implemented a custom interrupt handler. Here, however, we consider the latency and overhead induced by Linux's \gls{NMI} handling process negligible, hence we opted to keep the \gls{NMI} configuration here, but override \texttt{perf}'s handler and register our own during module initialization. This implies that, in the same way as when using our analysis framework, \texttt{perf}'s functionality is degraded when \textsc{avxfreq} is enabled.

Further, during initialization we configure a performance counter on each of the \gls{CPU}'s physical cores to count  \texttt{FP\_ARITH\_INST\_RETIRED.512B\_PACKED\_DOUBLE} events in both user-mode and kernel-mode and to trigger an interrupt as soon as a single assertion has occurred. These events are asserted every time a \SI{512}{\bit} packed double-precision vector \gls{microinst} has completed execution \cite{intelsdmsysprogguide}.

When our \gls{NMI} handler is called, and thus as soon as a vector instruction as selected above was executed, the handler sets the virtual license to level~1 and reconfigures the \gls{PMU} to count further assertions of the previously selected event without triggering more interrupts. Secondly, a \SI{100}{\micro\second} timer is configured using \gls{Linux}'s \texttt{hrtimer} subsystem which provides precise timers with nanosecond resolution for kernel-internal purposes. After the timer has elapsed, we read the previously configured performance counter to check whether any of the instructions we are looking for have been executed since the timer was started. If yes, and if the average throughput equals at least $1$ operation per cycle, we switch to virtual license level~2 -- this is to approximate the differing behaviors of the unrolled and non-unrolled versions as described in \Cref{sec:analysis:results:downclocking}. The performance counter is then reset and the timer is restarted to trigger again after another \SI{100}{\micro\second}. Otherwise, as long as none of the monitored \SI[number-unit-product=-]{512}{\bit} vector instructions were executed, we start counting the total elapsed time without them, again in \SI{100}{\micro\second} steps, and we reset the counter as soon as the performance counter rises above zero. Like in the first case, we reset the timer for another \SI{100}{\micro\second} unless \SI{600}{\micro\second} have already been elapsed -- then, we only wait for \SI{66}{\micro\second} to resemble the upclocking results from our analysis. As soon as this last timer is over we revert back to level~0 and reset to our initial state as configured during system boot so that we can run through the same procedure again when heavy instructions are being executed once more.

This way, \textsc{avxfreq} is capable of fulfilling what we outlined at the beginning of this section: we switch to frequencies equivalent to turbo license levels~1 and~2 according to the load and revert back to level~0 as soon as the load situation was alleviated for a continuous time period of about \SI[quotient-mode=fraction]{2/3}{\milli\second}. Note that we did not implement any artificial delays during the downclocking as we assume the delays we observed in our analysis to equal the time required by the hardware to complete the frequency reduction. Furthermore, the switch from level~1 to level~2 only occurs after a delay of \SI{100}{\micro\second} and not immediately as implemented in the hardware. This is because we use the throughput to decide whether to transition into level~2 -- and that requires some time for a meaningful measurement. We probably could achieve an improved approximation by using shorter delays at the expense of system performance. A possible trade-off could be to only make the first delay shorter. However, that remains future work for now.

In general, it should not be forgotten that this implementation only tries to reflect a subset of the results from our analysis in \Cref{sec:analysis}. Some parts are not possible to reimplement, e.g., because there are no performance events available specifically for integer vector instructions \cite{intelsdmsysprogguide}. For other parts, we deliberately decided not to implement them for now in order to reduce complexity -- what we built should be enough for a basic evaluation of our approach.

\begin{figure*}
	\centering
	\begin{tikzpicture}[font=\scriptsize]
		\sffamily
		\pgfmathsetmacro{\componentrectwidth}{4}
		\pgfmathsetmacro{\componentrectheight}{6.4}
		\pgfmathsetmacro{\separatordist}{0.5}
		\pgfmathsetmacro{\arrowoverlength}{0.25}
		\pgfmathsetmacro{\arrowlength}{2*(\arrowoverlength + \separatordist)}
		\pgfmathsetmacro{\halfcomponentrectwidth}{\componentrectwidth*0.5}
		\pgfmathsetmacro{\quartercomponentrectwidth}{\componentrectwidth*0.25}

		% kernel
		\draw (0cm,0cm) rectangle ++(\componentrectwidth cm,\componentrectheight) [ref=kernel-rect];
		\node[color=kitblue] at ([yshift=-0.4 cm] kernel-rect south) {\textsc{avxfreq}};

		% separator
		\draw[dashed, color=kitdarkgrey] ([xshift=\separatordist cm] kernel-rect south east) -- ([xshift=\separatordist cm] kernel-rect north east) [ref=separator];

		% user-space
		\draw ([xshift=\separatordist cm] separator south east) rectangle ++(\componentrectwidth,\componentrectheight) [ref=user-rect];
		\node[color=kitblue] at ([yshift=-0.4 cm] user-rect south) {User-space};

		% steps
		\node[align=center, anchor=north] at ([yshift=-0.1cm] kernel-rect north) {1. Setup PMU};

		\node[align=center, anchor=north] at ([yshift=-0.5cm] user-rect north) {2. Execute single \\ AVX instruction};

		\node[align=center, anchor=north, color=kityellow] at ([yshift=-1.2cm, xshift=-1.4cm] kernel-rect north) {\Huge\Lightning};
		\node[align=center, anchor=north] at ([yshift=-1.2cm] kernel-rect north) {3. PMU interrupt \\ Switch to level 1};

		\node[align=center, anchor=north] at ([yshift=-1.9cm] user-rect north) {4. Keep executing \\ AVX instructions};

		\node[align=center, anchor=north, color=kitdarkgrey] at ([yshift=-2.55cm, xshift=-1.4cm] kernel-rect north) {\small\Interval};
		\node[align=center, anchor=north] at ([yshift=-3cm, xshift=-1.4cm] kernel-rect north) {\tiny\SI{100}{\micro\second}};
		\node[align=center, anchor=north] at ([yshift=-2.6cm] kernel-rect north) {5. Timer interrupt \\ Switch to level 2};

		\node[align=center, anchor=north] at ([yshift=-3.3cm] user-rect north) {6. Execute AVX, \\ then scalar};

		\node[align=center, anchor=north, color=kitdarkgrey] at ([yshift=-3.95cm, xshift=-1.4cm] kernel-rect north) {\small\Interval};
		\node[align=center, anchor=north] at ([yshift=-4.4cm, xshift=-1.4cm] kernel-rect north) {\tiny\SI{100}{\micro\second}};
		\node[align=center, anchor=north] at ([yshift=-4cm] kernel-rect north) {7. Timer interrupt \\ Nothing happens};

		\node[align=center, anchor=north] at ([yshift=-4.7cm] user-rect north) {8. Keep executing \\ scalar instructions};

		\node[align=center, anchor=north, color=kitdarkgrey] at ([yshift=-5.35cm, xshift=-1.4cm] kernel-rect north) {\small\Interval};
		\node[align=center, anchor=north] at ([yshift=-5.8cm, xshift=-1.4cm] kernel-rect north) {\tiny\SI{666}{\micro\second}};
		\node[align=center, anchor=north] at ([yshift=-5.4cm] kernel-rect north) {9. Timer interrupt \\ Revert to level 0};
	\end{tikzpicture}
	\caption{Simplified exemplary license level switching process with \textsc{avxfreq} and a user-space process that executes \gls{AVX} instructions for a while before continuing with purely scalar instructions.}
	\label{fig:design:reimplementation:avxfreq}
\end{figure*}

\section{User-Space-Driven Decisions}
\label{sec:design:userspacedrivendecisions}

In the previous section we have described \textsc{avxfreq}, our reimplementation that implements a subset of the \gls{AVX} reclocking algorithm employed by Intel \glspl{CPU}. This work is intended to lay a foundation that may be used for building and evaluating possible optimizations of the \gls{DVFS} mechanism without needing to modify the hardware itself. In this section, we present the implementation of a simple approach where user-space is given control over the license levels, based on the idea that a program with a heterogeneous workload knows best when it is going to execute \gls{AVX} instructions and when it is in a scalar-only stage. This approach is especially useful as it is presumably capable of providing us with a theoretically optimal baseline for \gls{AVX} reclocking that can be used to evaluate the quality of other algorithms. For this purpose, we have extended \textsc{avxfreq} with a user-space interface consisting of multiple system calls:

\begin{itemize}
	\item \mintinline{c}{int avxfreq_is_enabled(void)} \\
		Returns 1 if \textsc{avxfreq} was enabled during system boot, 0 otherwise. Only when \textsc{avxfreq} is enabled, user-space may use the other system calls for managing software-based reclocking and license levels.
	\item \mintinline{c}{int avxfreq_set_reclocking(bool reclock)} \\
		Can be used to enable and disable \textsc{avxfreq}'s own reclocking mechanism. When disabled, user-space must take caution to always reduce frequency when required. Otherwise, system stability may be at risk. Note that it is enabled by default when \textsc{avxfreq} is enabled.
	\item \mintinline{c}{int avxfreq_set_license(unsigned char license)} \\
		Invokes a license level transition to the level passed in the \texttt{license} parameter (either 0, 1, or 2).
\end{itemize}

While this interface is very simple, it allows to do exactly what we wanted to: a user-space program may use these methods to change the applied license level at will, and thus, choose appropriate license levels when it knows what instructions it is going to use over periods of time.