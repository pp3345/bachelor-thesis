\chapter{Analysis}
\label{sec:analysis}

In order to be able to evaluate potential means of improving Intel's \gls{AVX} reclocking algorithm, we first need to obtain thorough knowledge of the algorithm as it is implemented in current Intel x86 \glspl{CPU}. We can then use this knowledge for the software-based reimplementation presented in \Cref{sec:design} and to understand the hardware-induced constraints Intel needs to keep within, which is in turn necessary for designing a feasible and implementable improved reclocking algorithm.

Intel regularly publishes optimization manuals~\cite{inteloptimizationmanual} intended for compiler developers and software engineers which contain a vague description of the mechanism used for deciding when to lower or raise the processor's frequency upon execution of \gls{AVX} instructions. Precisely, Intel defines three \textit{turbo license levels}, which designate frequency offsets for different instruction mix scenarios:

\begin{itemize}
	\item Level~0: only non-demanding (i.e., scalar, \gls{SSE}, \gls{AVX1} or light \gls{AVX2}) instructions are being executed; a core may run at its maximum turbo frequency. This is the default state.
	\item Level~1: active during the execution of heavy \gls{AVX2} and/or light \gls{AVX-512} instructions. The maximum frequency is lowered to a \gls{SKU}-specific value.
	\item Level~2: used for the execution of heavy \gls{AVX-512} instructions. The maximum frequency is lowered to a \gls{SKU}-specific value that is further below the frequency used in level~1.
\end{itemize}

Here, \enquote{heavy} instructions are defined to be floating-point, integer multiplication or integer \gls{FMA} operations. Given these license levels, Intel states that it may take up to \SI{500}{\micro\second} until the new frequency is applied and about \SI{2}{\milli\second} until a core reverts to level~0 after executing the last \enquote{heavy} instruction. Before the frequency is lowered, a core operates at \enquote{a lower peak capability}, however, Intel does not further specify what that exactly means. Intel hints that the license decisions are not solely bound to the instruction types as given in the level descriptions, but rather depend on the mix of instructions executed within a certain time window.

In this chapter we will describe the design of a framework for analyzing the actual behavior of an x86 processor during the execution of \gls{AVX} instructions. Afterwards, we will present and evaluate the results generated when executed on a system equipped with a modern Intel \gls{CPU}. Finally, we compare our findings to what Intel maintains in their specification and point out the differences.

\section{Methodology}

\section{Design}

Our analysis framework consists of a module for the Linux kernel as well as a user-space component which interact with each other and makes use of the \gls{PMU}, a unit commonly found in modern microprocessors that enables software to measure performance and bottlenecks on the hardware level. In the following sections, we will present the design and features of these components and describe how they contribute to our analysis purposes.

\subsection{Performance Monitoring Unit (PMU)}
Modern x86 \glspl{CPU} commonly feature a \gls{PMU} \cite{intelsdmsysprogguide} which exposes a set of \textit{performance counters}, which may be configured to count assertions of a large set of \textit{performance events}.

Precisely, we use version~3 of the x86 \textit{Architectural Performance Monitoring} facility, which features three \textit{fixed counters} per logical core that count retired instructions, cycles during which the core is not in a halt state and \glsunset{TSC}\gls{TSC} cycles in unhalted state, respectively. The \glsreset{TSC}\gls{TSC} is a simple counter found in current x86 \glspl{CPU} that increments steadily with a fixed frequency, independent of the core clock, thus making it suitable for measuring wall-clock time. In addition to the fixed counters, eight freely configurable counters are available per physical core (four per logical core when \gls{SMT} is enabled). These counters may be set to count any of the performance events available for a specific microarchitecture, e.g., most architectures define events for cache hits/misses, execution stalls or load on specific execution units.

Each counter is represented via a \gls{MSR} and also configured through one. More specifically, software may configure the event to count (non-fixed counters only) and when to count (e.g., in user mode (ring~$\geq$~1) and/or kernel mode (ring~0)). Additionally, the counter can be configured to trigger an interrupt when it overflows. By setting the counter to its maximum value less an offset, this can be used as a mechanism to generate notifications when a certain amount of events of a specific type has occurred. The interrupt vector used for delivery can be configured in the core's \gls{APIC}'s \gls{LVT}. Optionally, the \gls{PMU} may be instructed to freeze all counters at their current values as soon as an interrupt is triggered.

\subsection{Kernel Component}

\subsection{User-Space Component}

\subsection{Execution Modes}