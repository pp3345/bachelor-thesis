\chapter{Analysis}
\label{sec:analysis}

In order to be able to evaluate potential means of improving Intel's \gls{AVX} reclocking algorithm, we first need to obtain thorough knowledge of the algorithm as it is implemented in current Intel x86 \glspl{CPU}. We can then use this knowledge for the software-based reimplementation presented in \Cref{sec:design} and to understand the hardware-induced constraints Intel needs to keep within, which is in turn necessary for designing a feasible and implementable improved reclocking algorithm.

Intel regularly publishes optimization manuals~\cite{inteloptimizationmanual} intended for compiler developers and software engineers which contain a vague description of the mechanism used for deciding when to lower or raise the processor's frequency upon execution of \gls{AVX} instructions. Precisely, Intel defines three \textit{turbo license levels}, which designate frequency offsets for different instruction mix scenarios:

\begin{itemize}
	\item Level~0: only non-demanding (i.e., scalar, \gls{SSE}, \gls{AVX1} or light \gls{AVX2}) instructions are being executed; a core may run at its maximum turbo frequency. This is the default state.
	\item Level~1: active during the execution of heavy \gls{AVX2} and/or light \gls{AVX-512} instructions. The maximum frequency is lowered to a \gls{SKU}-specific value.
	\item Level~2: used for the execution of heavy \gls{AVX-512} instructions. The maximum frequency is lowered to a \gls{SKU}-specific value that is further below the frequency used in level~1.
\end{itemize}

Here, \enquote{heavy} instructions are defined to be floating-point, integer multiplication or integer \gls{FMA} operations. Given these license levels, Intel states that it may take up to \SI{500}{\micro\second} until the new frequency is applied and about \SI{2}{\milli\second} until a core reverts to level~0 after executing the last \enquote{heavy} instruction. Before the frequency is lowered, a core operates at \enquote{a lower peak capability}, however, Intel does not further specify what that exactly means. Intel hints that the license decisions are not solely bound to the instruction types as given in the level descriptions, but rather depend on the mix of instructions executed within a certain time window.

In this chapter we will describe the design of a framework that allows us to analyze the actual behavior of an x86 processor during the execution of \gls{AVX} instructions. Afterwards, we will present and evaluate the results generated when executed on a system equipped with a modern Intel \gls{CPU}. Finally, we compare our findings to what Intel maintains in their specification and point out deviations of the timings between the actual behavior and their claims.

\section{Methodology}
\label{sec:analysis:methodology}

For our reimplementation, our goal is to create a model of the reclocking behavior of an \gls{AVX-512}-capable \gls{CPU} that is as complete as possible and reflects the decisions made by the hardware with high accuracy. Therefore, by conducting this analysis, we want to answer the following aspects:

\begin{itemize}
	\item When exactly does a \gls{CPU} core decide to reduce or raise it's frequency during and after \gls{AVX} execution?
	\item How much time do turbo license level switches need?
	\item Do the \glspl{CPU} switch directly from level 0 to level 2 in case of heavy \gls{AVX-512} instructions or is there a step to level 1 in between?
	\item What does Intel mean by \enquote{lower peak capability} while lowering the clock?
	\item How complete is Intel's description of the reclocking algorithm?
\end{itemize}

In order to create a precise model we want to analyze these questions in different scenarios, i.e., for different instruction types, for different global load situations as well as with and without enabled turbo frequencies. To reach our goal, we run our analysis framework with synthetic code snippets that are designed to trigger the behavior to be analyzed.

\section{Design}
\label{sec:analysis:design}

Our analysis framework consists of a module for the Linux kernel as well as a user-space component which interact with each other and make use of the \gls{PMU}, a unit commonly found in modern microprocessors that enables software to measure performance and bottlenecks on the hardware level. In the following sections, we will present the design and features of these components and describe how they contribute to our analysis purposes.

\subsection{Performance Monitoring Unit (PMU)}
\label{sec:analysis:design:pmu}
Modern x86 \glspl{CPU} commonly feature a \gls{PMU} \cite{intelsdmsysprogguide} which exposes a set of \textit{performance counters} that may be configured to count assertions of a large set of \textit{performance events}.

Precisely, we use version~3 of the x86 \textit{Architectural Performance Monitoring} facility, which features three \textit{fixed counters} per logical core that count retired instructions, cycles during which the core is not in a halt state and \glsunset{TSC}\gls{TSC} cycles in unhalted state, respectively. The \glsreset{TSC}\gls{TSC} is a simple counter found in current x86 \glspl{CPU} that increments steadily with a fixed frequency, independent of the core clock, thus making it suitable for measuring wall-clock time. In addition to the fixed counters, eight freely configurable counters are available per physical core (four per logical core when \gls{SMT} is enabled). These counters may be set to count any of the performance events available for a specific microarchitecture, e.g., most architectures define events for cache hits/misses, execution stalls or load on specific execution units.

Each counter is represented via a \gls{MSR} and also configured through one. More specifically, software may configure the event to count (non-fixed counters only) and when to count (e.g., in user mode (ring~$\geq$~1) and/or kernel mode (ring~0)). Additionally, the counter can be configured to trigger an interrupt when it overflows. By setting the counter to its maximum value less an offset, this can be used as a mechanism to generate notifications when a certain amount of events of a specific type has occurred. The interrupt vector used for delivery can be configured in the core's \gls{APIC}'s \gls{LVT}. Optionally, the \gls{PMU} may be instructed to freeze all counters at their current values as soon as an interrupt is triggered.

\subsection{Overview}
\label{sec:analysis:design:overview}

The analysis tool presented here is made up of a kernel and a user-space component where the former provides the latter with means to configure the \gls{PMU} and efficient handling for interrupts generated by performance counter overflows.

As depicted in a simplified way in \Cref{fig:analysis:design:overview}, the user-space component spawns $n\in\mathbb{N}$ \textit{execution threads} and $[1,n]$ \textit{wait threads}, each corresponding to exactly one execution thread. Upon execution, each execution thread generates a \gls{PMU} configuration designed to produce the desired measurements, which is then applied by the kernel module. Now, the kernel module jumps back into user-space to an address previously defined by the execution thread which now starts to execute \gls{AVX} instructions until preempted by an overflow interrupt generated by the \gls{PMU} according to its configuration (as described in \Cref{sec:analysis:design:pmu}). Each wait thread is initially suspended until an interrupt is triggered on its corresponding execution thread, at which point it is resumed and provided with the raw performance counter values by the kernel component.

\begin{figure*}
	\centering
	\begin{tikzpicture}[font=\scriptsize]
		\sffamily
		\pgfmathsetmacro{\componentrectwidth}{4}
		\pgfmathsetmacro{\componentrectheight}{4.75}
		\pgfmathsetmacro{\separatordist}{2.75}
		\pgfmathsetmacro{\arrowoverlength}{0.25}
		\pgfmathsetmacro{\arrowlength}{2*(\arrowoverlength + \separatordist)}
		\pgfmathsetmacro{\halfcomponentrectwidth}{\componentrectwidth*0.5}
		\pgfmathsetmacro{\quartercomponentrectwidth}{\componentrectwidth*0.25}

		% kernel
		\draw (0cm,0cm) rectangle ++(\componentrectwidth cm,\componentrectheight) [ref=kernel-rect];
		\node[color=kitblue] at ([yshift=-0.4 cm] kernel-rect south) {Kernel module};
		%\draw[densely dotted] ([yshift=-1cm] kernel-rect north west) -- ++(\componentrectwidth,0) [ref=kernel-sep];
		%\node at ([yshift=-0.5cm] kernel-rect north) {Linux};

		% separator
		\draw[dashed, color=kitdarkgrey] ([xshift=\separatordist cm] kernel-rect south east) -- ([xshift=\separatordist cm] kernel-rect north east) [ref=separator];

		% user-space
		\draw ([xshift=\separatordist cm] separator south east) rectangle ++(\componentrectwidth,\componentrectheight) [ref=user-rect];
		\draw[densely dotted, color=kitdarkgrey] (user-rect south) -- ++(0,\componentrectheight);
		\node[align=center,anchor=north] at ([xshift=-\quartercomponentrectwidth cm] user-rect north) {execution \\ thread};
		\node[align=center,anchor=north] at ([xshift=+\quartercomponentrectwidth cm] user-rect north) {wait \\ thread};
		\node[color=kitblue] at ([yshift=-0.4 cm] user-rect south) {User-space component};

		% steps
		\draw[<-] ([yshift=-0.85cm,xshift=-\arrowoverlength cm] kernel-rect north east) -- ++(\arrowlength,0) node[align=center,pos=.5,above=0] {1. Configuration instructions};
		\node at ([yshift=-1.35cm] kernel-rect north) {2. Setup PMU};
		\draw[->] ([yshift=-2.0cm,xshift=-\arrowoverlength cm] kernel-rect north east) -- ++(\arrowlength,0) node[align=center,pos=.5,above=0] {3. Return to user-space};
		\node[align=center] at ([yshift=-2.5cm,xshift=-\quartercomponentrectwidth cm] user-rect north) {4. Execute \\ AVX};
		\node[color=kityellow] at ([yshift=-3.12cm,xshift=-1cm] kernel-rect north) {\Huge\Lightning};
		\node at ([yshift=-3.1cm] kernel-rect north) {5. Interrupt};
		\draw ([yshift=-3.75cm,xshift=-\arrowoverlength cm] kernel-rect north east) -- ++(\arrowlength,0) node[align=center,pos=.5,above=0] {6. PMU values};
		\draw[->] ([yshift=-3.75cm,xshift=\arrowoverlength cm] user-rect north west) -- ++(\halfcomponentrectwidth,0);
		\node[align=center] at ([yshift=-4.25cm,xshift=\quartercomponentrectwidth cm] user-rect north) {7. Process \\ results};
	\end{tikzpicture}
	\caption{Simplified analysis framework architecture. The kernel module enables the user-space component to configure the PMU and handles interrupts.}
	\label{fig:analysis:design:overview}
\end{figure*}

\subsection{Kernel Component}
\label{sec:analysis:design:kernel}

Our kernel component is not supposed to conduct any analysis tasks by itself, but is designed to aid the user-space component described later in \Cref{sec:analysis:design:userspace}. We chose to implement it as a module for version~5.1 of the Linux kernel, which implies that it is written in the C programming language. Existence and design of this kernel module are driven by our user-space component's needs to configure the \gls{PMU} in order to conduct the measurements required for our analysis. This can only be done from kernel-space.

During module startup, the \gls{PMU} is reset to a default state and the performance counter overflow interrupt vector is set in all core's \gls{APIC}'s \glspl{LVT}. Notably, this degrades the functionality of Linux's \texttt{perf} subsystem as parts of it rely on using the \gls{PMU}.

The module interfaces with user-space by defining a device class and then providing a virtual character device of the previously defined class, exposed via \texttt{/dev/reclocking\_analysis} in the virtual file system. User-space may then \texttt{open()} the provided device file and interact with the module by using several offered \texttt{ioctl()} calls.

Execution threads, on the one hand, initiate their execution by using the \texttt{SETUP ioctl()} call. A C \texttt{struct} must be passed that contains a set of \glspl{MSR} to be written by the kernel module -- these are used to configure the \gls{PMU}. Apart from that, a value for the \gls{instptr} may be passed that will be set in the thread's context before returning to user-space, thus, the thread will not directly return at the previous position in the \gls{libc}'s \texttt{ioctl()} wrapper but rather be redirected to another location in memory (this is used to directly jump into \gls{AVX} code). Optionally, the \texttt{r12}~x86 architectural register may also be set so that the code executed in user-space upon returning is able to access data structures in an easy manner without needing to use the stack. After applying the configuration, the \texttt{ioctl()} handler saves the current \gls{TSC} value and returns to user-space.

Wait threads, one the other hand, start with the \texttt{WAIT\_FOR\_INTERRUPT ioctl()}, which takes a pointer to a \texttt{interrupt\_result} structure in the user-space component where the resulting performance counter values shall be stored later as well as the numeric identifier of a \gls{CPU} core where an interrupt is expected to occur. The calling thread is suspended by setting it into \texttt{TASK\_UNINTERRUPTIBLE} state. This state in Linux's task state machine allows a thread to be waked only by the kernel itself and not via any user-space mechanisms (e.g., \glspl{UNIXsig}). Consequently, this way we ensure the execution flow is not interrupted unexpectedly.

It is expected that all execution threads that have a corresponding wait thread trigger a performance counter overflow interrupt some time after setup. The interrupt handler will then proceed with one of multiple actions as instructed by the \texttt{SETUP} call:

\begin{itemize}
	\item \texttt{WAKE\_WAIT\_THREAD}: this action reads all performance counters and writes them along with the current \gls{TSC} value and the recorded \gls{TSC} value at \texttt{SETUP} to the \texttt{interrupt\_result} structure of the corresponding wake wait, which is now waked up from suspension. The execution thread that triggered the interrupt is returned to its previous \acrlong{instptr} (before the \texttt{SETUP} call). Notably, from user-space's view, the \texttt{SETUP} call returns only now.
	\item \texttt{SET\_MSRS}: this is used for analysis tasks consisting of several consecutive steps. The \gls{TSC} value is recorded and another set of \glspl{MSR} is configured on the thread's core.
	\item \texttt{GOTO}: exactly like \texttt{SET\_MSRS}, but also sets a new \acrlong{instptr} on the execution thread.
\end{itemize}

Finally, the \gls{PMU}'s overflow bit and the \gls{APIC}'s state are reset in order to be ready for further interrupts.

\subsection{User-Space Component}
\label{sec:analysis:design:userspace}

\subsection{Execution Modes}
\label{sec:analysis:design:executionmodes}