\makeglossaries

\newacronym[description={An instruction set extension for x86 \glspl{CPU} which adds complex \glspl{vectorinst}}]{AVX}{AVX}{Advanced Vector Extensions}
\newacronym[description={The main execution unit (processor) of a computer}]{CPU}{CPU}{Central Processing Unit}\glsunset{CPU}
\newacronym[description={An instruction that multiplies two values and adds (or subtracts) a third one, i.e., $a*b+c$}]{FMA}{FMA}{fused multiply-add}
\newacronym[description={A specific processor model brought to market}]{SKU}{SKU}{Stock-Keeping Unit}\glsunset{SKU}
\newacronym[description={Various early \gls{vectorinst} sets for the x86 architecture, which are still commonly supported by modern \glspl{CPU} but do not require frequency reduction for stable execution}]{SSE}{SSE}{Streaming \gls{SIMD} Extensions}\glsunset{SSE}
\newacronym[description={See \gls{vectorinst}}]{SIMD}{SIMD}{Single Instruction, Multiple Data}
\newacronym[description={A unit found in many modern microprocessors that enables software to measure performance and bottlenecks on the hardware level}]{PMU}{PMU}{Performance Monitoring Unit}
\newacronym[description={A simple counter on current x86 processors that increments steadily with a fixed frequency and that can thus be used to measure wall-clock time}]{TSC}{TSC}{Time-Stamp Counter}
\newacronym[description={SMT-capable processors feed the execution pipeline of a single physical core simultaneously with instruction streams from multiple threads to achieve better execution unit utilization by presenting the core in the form of multiple \textit{logical cores}. x86 \glspl{CPU} often support twofold SMT}]{SMT}{SMT}{Simultaneous Multi-Threading}
\newacronym[description={A special register, usually for configuration or measurement purposes, that isn't defined by the instruction set architecture but is specific to a microarchitecture or \gls{SKU}}]{MSR}{MSR}{Model-Specific Register}
\newacronym[description={One of the interrupt controllers found in modern x86 \glspl{CPU}}]{APIC}{APIC}{Advanced Programmable Interrupt Controller}\glsunset{APIC}
\newacronym[description={An \gls{APIC}'s LVT stores the interrupt vectors to use for certain purposes}]{LVT}{LVT}{Local Vector Table}
\newacronym[description={The instruction pointer of a thread contains the memory address of the instruction that is currently being executed}]{instptr}{IP}{instruction pointer}

\newglossaryentry{AVX1}{name={AVX1},description={The initial version of \gls{AVX}}}
\newglossaryentry{AVX2}{name={AVX2},description={\gls{AVX} version 2, which added new instructions and \SI{256}{\bit}-wide vectors}}
\newglossaryentry{AVX-512}{name={AVX-512},description={The third iteration of the \gls{AVX} instruction set with support for \SI{512}{\bit}-wide vectors}}
\newglossaryentry{vectorinst}{name={vector instruction},description={A vector instruction is an instruction for a microprocessor which executes an operation not on only just one value, but on a vector consisting of several values}}
\newglossaryentry{libc}{name={libc},description={libc is the name of the standard library of the C programming language}}
\newglossaryentry{UNIXsig}{name={UNIX signal},description={Signals defined by the UNIX operation system specification that may be sent to a process either by another process or by the kernel itself. Most signals terminate a process by default, unless the process opted to use custom handling}}