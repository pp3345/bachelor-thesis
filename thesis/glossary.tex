\makeglossaries

\newacronym[description={An instruction set extension for x86 \glspl{CPU} which adds complex \glspl{vectorinst}}]{AVX}{AVX}{Advanced Vector Extensions}
\newacronym[description={The main execution unit (processor) of a computer}]{CPU}{CPU}{Central Processing Unit}\glsunset{CPU}
\newacronym[description={An instruction that multiplies two values and adds (or subtracts) a third one, i.e., $a*b+c$}]{FMA}{FMA}{fused multiply-add}
\newacronym[description={A specific processor model brought to market}]{SKU}{SKU}{Stock-Keeping Unit}\glsunset{SKU}
\newacronym[description={Various early \gls{vectorinst} sets for the x86 architecture, which are still commonly supported by modern \glspl{CPU} but do not require frequency reduction for stable execution}]{SSE}{SSE}{Streaming SIMD Extensions}
\newacronym[description={See \gls{vectorinst}}]{SIMD}{SIMD}{Single Instruction, Multiple Data}
\newacronym[description={A unit found in many modern microprocessors that enables software to measure performance and bottlenecks on the hardware level}]{PMU}{PMU}{Performance Monitoring Unit}
\newacronym[description={A simple counter on current x86 processors that increments steadily with a fixed frequency and that can thus be used to measure wall-clock time}]{TSC}{TSC}{Time-Stamp Counter}
\newacronym[description={SMT-capable processors feed the execution pipeline of a single physical core simultaneously with instruction streams from multiple threads to achieve better execution unit utilization by presenting the core in the form of multiple \textit{logical cores}. x86 \glspl{CPU} often support twofold SMT}]{SMT}{SMT}{Simultaneous Multi-Threading}
\newacronym[description={A special register, usually for configuration or measurement purposes, that isn't defined by the instruction set architecture but is specific to a microarchitecture or \gls{SKU}}]{MSR}{MSR}{model-specific register}
\newacronym[description={One of the interrupt controllers found in modern x86 \glspl{CPU}}]{APIC}{APIC}{Advanced Programmable Interrupt Controller}\glsunset{APIC}
\newacronym[description={An \gls{APIC}'s LVT stores the interrupt vectors to use for certain purposes}]{LVT}{LVT}{Local Vector Table}
\newacronym[description={The instruction pointer of a thread contains the memory address of the instruction that is currently being executed}]{instptr}{IP}{instruction pointer}
\newacronym[description={The standard file format for executable binary files commonly used by many UNIX systems, including \gls{Linux}}]{ELF}{ELF}{Executable and Linkable Format}\glsunset{ELF}
\newacronym[description={The default scheduler used in current \gls{Linux} kernels. Designed to offer both low latency for interactive systems and high throughput for servers}]{CFS}{CFS}{Completely Fair Scheduler}
\newacronym[description={A processor featuring out-of-order support is capable of executing instructions in a different order than given in the program while maintaining correct and consistent results in order to achieve higher utilization of a \gls{superscalar}}]{OoO}{OoO}{out-of-order}
\newacronym[description={Different manufacturers have used different definitions over time. For current-generation Intel processors, this value is a hard cap for the chip's power consumption}]{TDP}{TDP}{Thermal Design Power}
\newacronym[description={In general, ASLR is a technique that tries to prevent malicious code from being able to guess the memory locations of symbols or data in a process by randomizing their positions. KASLR implements this for the \gls{Linux} kernel}]{KASLR}{KASLR}{Kernel Address Space Layout Randomization}
\newacronym[description={NVMe is a modern device interface specification designed specifically for \glspl{SSD} attached via \gls{PCIe}, crafted to reach lower latencies and higher throughput compared to earlier interfaces}]{NVMe}{NVMe}{Non-Volatile Memory Express}\glsunset{NVMe}
\newacronym[description={PCIe is a high-speed bus for connecting peripheral devices (e.g., storage, graphics processors, \dots) with the rest of a computer system}]{PCIe}{PCIe}{Peripheral Component Interconnect Express}\glsunset{PCIe}
\newacronym[description={Average amount of instructions executed by a processor per cycle for a given workload}]{IPC}{IPC}{instructions per cycle}
\newacronym[description={The modern successor of the classical \gls{BIOS} with features like more thorough graphics and networking support and a cryptographically secured operating system boot process}]{UEFI}{UEFI}{Unified Extensible Firmware Interface}\glsunset{UEFI}
\newacronym[description={The first software loaded upon starting a legacy x86-based computer system, which conducts early initialization work before booting the operating system. Today, however, BIOSes are considered deprecated and have mostly been replaced with \glspl{UEFI}}]{BIOS}{BIOS}{Basic Input/Output System}\glsunset{BIOS}
\newacronym[description={A unit in current Intel processors that monitors load and power consumption of different components within the chip and dynamically assigns frequencies and voltages as needed}]{PCU}{PCU}{Power Control Unit}
\newacronym[description={A generic term for techniques that seek to maximize both energy efficiency and performance by dynamically selecting frequencies along with appropriate voltages for different components of a microprocessor}]{DVFS}{DVFS}{Dynamic Voltage and Frequency Scaling}
\newacronym[description={A performance state describes at the very least a specific frequency. Depending on the specific hardware, it may also include a voltage corresponding to the frequency}]{P-state}{P-state}{performance state}\glsunset{P-state}
\newacronym[description={A feature of modern Intel \glspl{CPU} where a \gls{PCU} in the chip completely takes over \gls{P-state} management}]{HWP}{HWP}{Hardware-Controlled Performance States}\glsunset{HWP}
\newacronym[description={An interrupt that can not be disabled and is always immediately delivered}]{NMI}{NMI}{non-maskable interrupt}
\newacronym[description={A kind of processor that is specifically designed for 3D graphics purposes. Nowadays, GPUs are often also employed for scientific computing and artificial intelligence}]{GPU}{GPU}{Graphics Processing Unit}\glsunset{GPU}
\newacronym[description={A very early \gls{vectorinst} set extension to x86, introduced by Intel in 1997}]{MMX}{MMX}{Multi Media Extension}
\newacronym[description={MOSFETs are an important type of semiconductor transistors that are capable of changing their conductivity depending on the voltage applied to their gate. Silicon MOSFETs are commonly used in micro- and nanoprocessors}]{MOSFET}{MOSFET}{Metal-Oxide-Semiconductor Field-Effect Transistor}\glsunset{MOSFET}
\newacronym[description={TLS is a wide-spread network protocol used for the secure encryption and authentication of internet traffic}]{TLS}{TLS}{Transport Layer Security}
\newacronym[description={CMOS technology implements digital logic gatters by using pairs of positively charged (pMOS) and negatively charged (nMOS) \glspl{MOSFET} and is commonly used to construct integrated circuits}]{CMOS}{CMOS}{Complementary Metal-Oxide-Semiconductor}\glsunset{CMOS}
\newacronym[description={ACPI is a standard that defines means for discovery and configuration of hardware devices as well as system power management. ACPI is commonly implemented by all modern \glspl{UEFI} and operating systems}]{ACPI}{ACPI}{Advanced Configuration and Power Interface}\glsunset{ACPI}
\newacronym[description={MuQSS is an alternative scheduler developed by Con Kolivas for the \gls{Linux} kernel that is not part of the official \gls{Linux} packages. One of its primary aims is to provide strong latency guarantees for interactive workloads}]{MuQSS}{MuQSS}{Multiple Queue Skiplist Scheduler}
\newacronym[description={PEBS is a feature of current Intel processors that builds atop the \gls{PMU} and allows to record precise data including register states and time-stamps upon the assertion of performance events}]{PEBS}{PEBS}{Processor Event-Based Sampling}
\newacronym[description={Hard disks are mass storage devices that use magnetization to store data on rotating disks. Nowadays, HDDs increasingly vanish from all kinds of computers and are more commonly replaced by \glspl{SSD}}]{HDD}{HDD}{hard disk drive}
\newacronym[description={A solid state drive uses flash memory for persistent data storage. Modern SSDs commonly beat the performance of \glspl{HDD} by several orders of magnitude with only a fraction of their energy consumption}]{SSD}{SSD}{solid-state drive}\glsunset{SSD}

\newglossaryentry{AVX1}{name={AVX1},description={The initial version of \gls{AVX} with \SI{256}{\bit}-wide vectors}}
\newglossaryentry{AVX2}{name={AVX2},description={\gls{AVX} version 2, which extended the previous \gls{AVX1} with new instructions}}
\newglossaryentry{AVX-512}{name={AVX-512},description={The third iteration of the \gls{AVX} instruction set with support for \SI{512}{\bit}-wide vectors}}
\newglossaryentry{vectorinst}{name={vector instruction},description={A vector instruction is an instruction for a microprocessor which executes an operation not on only just one value, but on a vector consisting of several values}}
\newglossaryentry{libc}{name={libc},description={libc is the name of the standard library of the C programming language}}
\newglossaryentry{UNIXsig}{name={UNIX signal},description={Signals defined by the UNIX operating system specification that may be sent to a process either by another process or by the kernel itself. Most signals terminate a process by default, unless the process opted to use custom handling}}
\newglossaryentry{instprefetcher}{name={instruction prefetcher},description={A unit in a microprocessor that predictively loads instructions from system memory before they are executed in order to prevent pipeline stalls}}
\newglossaryentry{L3}{name={L3 cache},description={Level 3 cache, usually the last cache in the hierarchy on most modern \glspl{CPU}. Typically several Megabytes large and shared between all cores}}
\newglossaryentry{macroinst}{name={macro-instruction},description={A macro-instruction is an instruction as defined by the instruction set architecture and may be split up into several \glspl{microinst} for execution by a specific implementation (i.e., a processor)}}
\newglossaryentry{microinst}{name={micro-instruction},description={Micro-instructions are the instructions passed into the execution engine of a processor after being split up from the \glspl{macroinst} that make up the program that is being executed}}
\newglossaryentry{moore}{name={Moore's Law}, description={Moore's Law is named after Gordon Moore, one of the co-founders of Intel. It states that the integration density of integrated circuits doubles every two years}}
\newglossaryentry{dennard}{name={Dennard's Law},description={Dennard's Law was devised by Robert Dennard and states that the power density of transistors stays constant as they become smaller}}
\newglossaryentry{Linux}{name={Linux},description={Linux is an operating system kernel originally published by Linus Torvalds in 1991. Today it dominates many markets, including smartphones, servers, and high-performance computing and has become the world's most widely used kernel. It even runs on cars, trains, and airplanes}}
\newglossaryentry{x86}{name={x86},description={x86 is an instruction set architecture introduced by Intel in 1978. Today, x86-based processors are commonly found in workstations, servers, and laptops}}
\newglossaryentry{superscalar}{name={superscalar pipeline},description={Processors featuring a superscalar pipeline are capable of executing multiple instruction of the same thread in parallel by having duplicates of all necessary ressources (e.g., decode and execution units)}}