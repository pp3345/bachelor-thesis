\section{Analysis}

\begin{frame}[t]{Methodology}
	We want to find out
	\begin{itemize}
		\item when exactly a core will reduce or raise its frequency,
		\item how long it needs to do that,
		\item and whether Intel's description is correct and complete.
	\end{itemize}
	\pause
	How can we measure that?
	\begin{itemize}
		\item Use performance counters
		\item Run synthetic code snippets designed to trigger specific behavior
	\end{itemize}
\end{frame}

\begin{frame}[t]{Framework}
	\begin{itemize}
		\item Linux kernel module
		\begin{itemize}
			\item configures Performance Monitoring Unit (PMU)
			\item handles interrupts
		\end{itemize}
		\item User-space program
		\begin{itemize}
			\item instructs kernel component
			\item conducts measurements
		\end{itemize}
	\end{itemize}
\end{frame}

\begin{frame}{Measurement Modes \\ \small Downclocking}
	\begin{figure}
		\centering
		\begin{subfigure}[b]{0.5\textwidth}
			\begin{tikzpicture}[font=\scriptsize]
			\sffamily
			\pgfmathsetmacro{\levelheight}{0.2}
			\pgfmathsetmacro{\linelength}{3}
			\pgfmathsetmacro{\barlength}{0.3}
			
			% legend
			\draw[fill=kitgreen, color=kitgreen] (-2.2, 0.9) rectangle ++(\barlength cm, \levelheight cm);
			\node[minimum height=\levelheight cm, text centered, anchor=west] at (-1.85, 1) {AVX};
			
			%\node[color=kityellow, align=center] at (-2.05, 0.3) {\large\Lightning};
			%\node[minimum height=\levelheight cm, text centered, anchor=west] at (-1.85, 0.3) {Interrupt};
			
			% level 0
			\node[color=kitblue, minimum height=\levelheight cm, text centered] at (0, 1) {Level~0};
			\draw[color=kitdarkgrey] (0.7, 1) -- ++(\linelength cm,0);
			
			% level 1
			\node[color=kitblue, minimum height=\levelheight cm, text centered] at (0, 0.3) {Level~1};
			\draw[color=kitdarkgrey] (0.7, 0.3) -- ++(\linelength cm,0);
			
			% bars and transitions
			\draw[<-] (0.8, 1.1) -- ++(0, 0.2);
			\draw[fill=kitgreen, color=kitgreen] (0.8, 0.9) rectangle ++(2cm, \levelheight cm);
			\draw[->, anchor=east] (2.8, 0.9) -- ++(0, -0.6);% node[pos=0.5, anchor=west] {\large\color{kityellow}\Lightning};
			
			\draw[decorate, decoration={brace, amplitude=5pt, mirror}] (0.8, 0.2) -- ++(2cm,0) node[pos=0.5, anchor=north, yshift=-0.1cm, align=center] {measure \\ time};
			\end{tikzpicture}
			\caption{Level~1}
		\end{subfigure}%
		\begin{subfigure}[b]{0.5\textwidth}
			\begin{tikzpicture}[font=\scriptsize]
			\sffamily
			\pgfmathsetmacro{\levelheight}{0.2}
			\pgfmathsetmacro{\linelength}{5}
			
			% level 0
			\node[color=kitblue, minimum height=\levelheight cm, text centered] at (0, 1) {Level~0};
			\draw[color=kitdarkgrey] (0.7, 1) -- ++(\linelength cm,0);
			
			% level 1
			\node[color=kitblue, minimum height=\levelheight cm, text centered] at (0, 0.3) {Level~1};
			\draw[color=kitdarkgrey] (0.7, 0.3) -- ++(\linelength cm,0);
			
			% level 2
			\node[color=kitblue, minimum height=\levelheight cm, text centered] at (0, -0.4) {Level~2};
			\draw[color=kitdarkgrey] (0.7, -0.4) -- ++(\linelength cm,0);
			
			% bars and transitions
			\draw[<-] (0.8, 1.1) -- ++(0, 0.2);
			\draw[fill=kitgreen, color=kitgreen] (0.8, 0.9) rectangle ++(2cm, \levelheight cm);
			\draw[->, anchor=east] (2.8, 0.9) -- ++(0, -0.5);
			\draw[fill=kitgreen, color=kitgreen] (2.8, 0.2) rectangle ++(2cm, \levelheight cm);
			\draw[->, anchor=east] (4.8, 0.2) -- ++(0, -0.6);% node[pos=0.5, anchor=west] {\large\color{kityellow}\Lightning};
			
			\draw[decorate, decoration={brace, amplitude=5pt, mirror}] (0.8, -0.5) -- ++(4cm,0) node[pos=0.5, anchor=north, yshift=-0.1cm, align=center] {measure \\ time};
			\end{tikzpicture}
			\caption{Level~2}
		\end{subfigure}
	\end{figure}
\end{frame}

\begin{frame}{Measurement Modes \\ \small Upclocking}
	\begin{figure}[t]
		\centering
		\begin{subfigure}[b]{0.5\textwidth}
			\begin{tikzpicture}[font=\scriptsize]
			\sffamily
			\pgfmathsetmacro{\levelheight}{0.2}
			\pgfmathsetmacro{\linelength}{3}
			\pgfmathsetmacro{\legendbarlength}{0.3}
			\pgfmathsetmacro{\barlength}{1.2}
			
			% legend
			\draw[fill=kitgreen, color=kitgreen] (-2.2, 1.6) rectangle ++(\legendbarlength cm, \levelheight cm);
			\node[minimum height=\levelheight cm, text centered, anchor=west] at (-1.85, 1.7) {AVX};
			
			\draw[fill=kitbrown, color=kitbrown] (-2.2, 0.9) rectangle ++(\legendbarlength cm, \levelheight cm);
			\node[minimum height=\levelheight cm, text centered, anchor=west] at (-1.85, 1) {Scalar};
			
			%\node[color=kityellow, align=center] at (-2.05, 0.3) {\large\Lightning};
			%\node[minimum height=\levelheight cm, text centered, anchor=west] at (-1.85, 0.3) {Interrupt};
			
			% level 0
			\node[color=kitblue, minimum height=\levelheight cm, text centered] at (0, 1) {Level~0};
			\draw[color=kitdarkgrey] (0.6, 1) -- ++(\linelength cm,0);
			
			% level 1
			\node[color=kitblue, minimum height=\levelheight cm, text centered] at (0, 0.3) {Level~1};
			\draw[color=kitdarkgrey] (0.6, 0.3) -- ++(\linelength cm,0);
			
			% bars and transitions
			\draw[<-] (0.7, 1.1) -- ++(0, 0.2);
			\draw[fill=kitgreen, color=kitgreen] (0.7, 0.9) rectangle ++(\barlength cm, \levelheight cm);
			\draw[->] (1.9, 0.9) -- ++(0, -0.5);% node[pos=0.5, anchor=west] {\large\color{kityellow}\Lightning};
			\draw[fill=kitbrown, color=kitbrown] (1.9, 0.2) rectangle ++(\barlength cm, \levelheight cm);
			\draw[->] (3.1, 0.4) -- ++(0, 0.6);% node[pos=0.41666, anchor=west] {\large\color{kityellow}\Lightning};
			
			\draw[decorate, decoration={brace, amplitude=5pt, mirror}] (1.9, 0.2) -- ++(\barlength cm,0) node[pos=0.5, anchor=north, yshift=-0.1cm, align=center] {measure \\ time};
			\end{tikzpicture}
			\caption{Level~1}
		\end{subfigure}%
		\begin{subfigure}[b]{0.5\textwidth}
			\begin{tikzpicture}[font=\scriptsize]
			\sffamily
			\pgfmathsetmacro{\linelength}{4.2}
			\pgfmathsetmacro{\barlength}{1.2}
			\pgfmathsetmacro{\barheight}{0.2}
			\pgfmathsetmacro{\linemargin}{0.6}
			\pgfmathsetmacro{\linepadding}{0.1}
			\pgfmathsetmacro{\linedistance}{0.7}
			\pgfmathsetmacro{\linezeroy}{1}
			\pgfmathsetmacro{\lineoney}{\linezeroy-\linedistance}
			\pgfmathsetmacro{\linetwoy}{\lineoney-\linedistance}
			\pgfmathsetmacro{\interbararrowheight}{\linedistance-\barheight}
			\pgfmathsetmacro{\measurey}{\linetwoy-0.5*\barheight}
			
			% level 0
			\node[color=kitblue, minimum height=\barheight cm, text centered] at (0, \linezeroy) {Level~0};
			\draw[color=kitdarkgrey] (\linemargin, \linezeroy) -- ++(\linelength cm,0);
			
			% level 1
			\node[color=kitblue, minimum height=\barheight cm, text centered] at (0, \lineoney) {Level~1};
			\draw[color=kitdarkgrey] (\linemargin, \lineoney) -- ++(\linelength cm,0);
			
			% level 2
			\node[color=kitblue, minimum height=\barheight cm, text centered] at (0, \linetwoy) {Level~2};
			\draw[color=kitdarkgrey] (\linemargin, \linetwoy) -- ++(\linelength cm,0);
			
			% bars and transitions
			\pgfmathsetmacro{\baronex}{\linemargin+\linepadding}
			\pgfmathsetmacro{\baroney}{\linezeroy-0.5*\barheight}
			\draw[fill=kitgreen, color=kitgreen] (\baronex, \baroney) rectangle ++(\barlength cm, \barheight cm);
			
			\pgfmathsetmacro{\bartwox}{\baronex+\barlength}
			\pgfmathsetmacro{\bartwoy}{\lineoney-0.5*\barheight}
			\draw[fill=kitgreen, color=kitgreen] (\bartwox, \bartwoy) rectangle ++(\barlength cm, \barheight cm);
			
			\pgfmathsetmacro{\barthreex}{\bartwox+\barlength}
			\pgfmathsetmacro{\barthreey}{\linetwoy-0.5*\barheight}
			\draw[fill=kitbrown, color=kitbrown] (\barthreex, \barthreey) rectangle ++(\barlength cm, \barheight cm);
			
			\pgfmathsetmacro{\arrowoney}{\baroney+\barheight}
			\draw[<-] (\baronex, \arrowoney) -- ++(0, 0.2);
			
			\draw[->, anchor=east] (\bartwox, \baroney) -- ++(0, -\interbararrowheight cm);
			
			\draw[->, anchor=east] (\barthreex, \bartwoy) -- ++(0, -\interbararrowheight cm);% node[pos=0.5, anchor=west] {\large\color{kityellow}\Lightning};
			
			\pgfmathsetmacro{\arrowfourx}{\barthreex+\barlength}
			\pgfmathsetmacro{\arrowfoury}{\barthreey+\barheight}
			\pgfmathsetmacro{\arrowfourlength}{\interbararrowheight+0.5*\barheight+\linedistance}
			\pgfmathsetmacro{\arrowfournodepos}{0.5*\interbararrowheight/\arrowfourlength}
			\draw[->, anchor=east] (\arrowfourx,\arrowfoury) -- ++(0, \arrowfourlength cm);% node[pos=\arrowfournodepos, anchor=west] {\large\color{kityellow}\Lightning};
			
			\draw[decorate, decoration={brace, amplitude=5pt, mirror}] (\barthreex, \measurey) -- ++(\barlength cm,0) node[pos=0.5, anchor=north, yshift=-0.1cm, align=center] {measure \\ time};
			\end{tikzpicture}
			\caption{Level~2}
		\end{subfigure}
	\end{figure}
\end{frame}

\begin{frame}{Measurement Modes \\ \small Throttle Activation}
	\begin{figure}
		\centering
		\begin{tikzpicture}[font=\scriptsize]
		\sffamily
		\pgfmathsetmacro{\levelheight}{0.2}
		\pgfmathsetmacro{\linelength}{1.8}
		\pgfmathsetmacro{\legendbarlength}{0.3}
		\pgfmathsetmacro{\barlength}{1.2}
		
		% legend
		\draw[fill=kitgreen, color=kitgreen] (-2.5, 0.9) rectangle ++(\legendbarlength cm, \levelheight cm);
		\node[minimum height=\levelheight cm, text centered, anchor=west] at (-2.15, 1) {AVX};
		
		%\node[color=kityellow, align=center] at (-2.35, 0.3) {\large\Lightning};
		%\node[minimum height=\levelheight cm, text centered, anchor=west] at (-2.15, 0.3) {Interrupt};
		
		% level 0
		\node[color=kitblue, minimum height=\levelheight cm, text centered] at (0, 1) {Level~0};
		\draw[color=kitdarkgrey] (0.6, 1) -- ++(\linelength cm,0);
		
		% throttle
		\node[color=kitblue, minimum height=\levelheight cm, text centered] at (0, 0.3) {Throttle};
		\draw[color=kitdarkgrey] (0.6, 0.3) -- ++(\linelength cm,0);
		
		% bars and transitions
		\draw[<-] (0.7, 1.1) -- ++(0, 0.2);
		\draw[fill=kitgreen, color=kitgreen] (0.7, 0.9) rectangle ++(\barlength cm, \levelheight cm);
		\draw[->] (1.9, 0.9) -- ++(0, -0.6);% node[pos=0.5, anchor=west] {\large\color{kityellow}\Lightning};
		
		\draw[decorate, decoration={brace, amplitude=5pt, mirror}] (0.7, 0.2) -- ++(\barlength cm,0) node[pos=0.5, anchor=north, yshift=-0.1cm, align=center] {measure \\ instructions};
		\end{tikzpicture}
	\end{figure}
\end{frame}

\begin{frame}{Measurement Modes \\ \small Trigger Instructions}
	\begin{figure}
		\centering
		\begin{subfigure}[b]{0.5\textwidth}
			\begin{tikzpicture}[font=\scriptsize]
			\sffamily
			\pgfmathsetmacro{\levelheight}{0.2}
			\pgfmathsetmacro{\linelength}{3}
			\pgfmathsetmacro{\legendbarlength}{0.3}
			\pgfmathsetmacro{\barlength}{1.5}
			
			% legend
			\draw[fill=kitgreen, color=kitgreen] (-2.2, 1.6) rectangle ++(\legendbarlength cm, \levelheight cm);
			\node[minimum height=\levelheight cm, text centered, anchor=west] at (-1.85, 1.7) {AVX};
			
			\draw[fill=kitbrown, color=kitbrown] (-2.2, 0.9) rectangle ++(\legendbarlength cm, \levelheight cm);
			\node[minimum height=\levelheight cm, text centered, anchor=west] at (-1.85, 1) {Scalar};
			
			%\node[color=kityellow, align=center] at (-2.05, 0.3) {\large\Lightning};
			%\node[minimum height=\levelheight cm, text centered, anchor=west] at (-1.85, 0.3) {Interrupt};
			
			% level 0
			\node[color=kitblue, minimum height=\levelheight cm, text centered] at (0, 1) {Level~0};
			\draw[color=kitdarkgrey] (0.6, 1) -- ++(\linelength cm,0);
			
			% level 1
			\node[color=kitblue, minimum height=\levelheight cm, text centered] at (0, 0.3) {Level~1};
			\draw[color=kitdarkgrey] (0.6, 0.3) -- ++(\linelength cm,0);
			
			% bars and transitions
			\draw[<-] (0.7, 1.1) -- ++(0, 0.2);
			\draw[fill=kitgreen, color=kitgreen] (0.7, 0.9) rectangle ++(0.6 cm, \levelheight cm);
			\draw[fill=kitbrown, color=kitbrown] (1.3, 0.9) rectangle ++(0.9 cm, \levelheight cm);
			\draw[->] (2.2, 0.9) -- ++(0, -0.6);% node[pos=0.5, anchor=west] {\large\color{kityellow}\Lightning};
			\draw[draw=none] (0.7, 0.9) rectangle ++(\barlength cm, \levelheight cm);% node[pos=0.5, align=center, color=black, font=\tiny] {\texttt{LOOP\_R12\_CMP}};
			
			\draw[decorate, decoration={brace, amplitude=3pt, mirror}] (0.7, 0.2) -- ++(0.6 cm,0) node[pos=0.5, anchor=north, yshift=-0.1cm, align=center] {measure \\ instructions};
			\end{tikzpicture}
			\caption{Level~1}
		\end{subfigure}%
		\begin{subfigure}[b]{0.5\textwidth}
			\begin{tikzpicture}[font=\scriptsize]
			\sffamily
			\pgfmathsetmacro{\linelength}{4.2}
			\pgfmathsetmacro{\barlength}{1.5}
			\pgfmathsetmacro{\barheight}{0.2}
			\pgfmathsetmacro{\linemargin}{0.6}
			\pgfmathsetmacro{\linepadding}{0.1}
			\pgfmathsetmacro{\linedistance}{0.7}
			\pgfmathsetmacro{\linezeroy}{1}
			\pgfmathsetmacro{\lineoney}{\linezeroy-\linedistance}
			\pgfmathsetmacro{\linetwoy}{\lineoney-\linedistance}
			\pgfmathsetmacro{\interbararrowheight}{\linedistance-\barheight}
			\pgfmathsetmacro{\measurey}{\linetwoy-0.5*\barheight}
			
			% level 0
			\node[color=kitblue, minimum height=\barheight cm, text centered] at (0, \linezeroy) {Level~0};
			\draw[color=kitdarkgrey] (\linemargin, \linezeroy) -- ++(\linelength cm,0);
			
			% level 1
			\node[color=kitblue, minimum height=\barheight cm, text centered] at (0, \lineoney) {Level~1};
			\draw[color=kitdarkgrey] (\linemargin, \lineoney) -- ++(\linelength cm,0);
			
			% level 2
			\node[color=kitblue, minimum height=\barheight cm, text centered] at (0, \linetwoy) {Level~2};
			\draw[color=kitdarkgrey] (\linemargin, \linetwoy) -- ++(\linelength cm,0);
			
			% bars and transitions
			\pgfmathsetmacro{\baronex}{\linemargin+\linepadding}
			\pgfmathsetmacro{\baroney}{\linezeroy-0.5*\barheight}
			\draw[fill=kitgreen, color=kitgreen] (\baronex, \baroney) rectangle ++(\barlength cm, \barheight cm);% node[pos=0.5, align=center, color=black, font=\tiny] {\texttt{LOOP\_AVX}};
			
			\pgfmathsetmacro{\bartwox}{\baronex+\barlength}
			\pgfmathsetmacro{\bartwoy}{\lineoney-0.5*\barheight}
			\draw[fill=kitgreen, color=kitgreen] (\bartwox, \bartwoy) rectangle ++(0.6 cm, \barheight cm);
			\draw[fill=kitbrown, color=kitbrown] (\bartwox+0.6, \bartwoy) rectangle ++(0.9 cm, \barheight cm);
			\draw[draw=none] (\bartwox, \bartwoy) rectangle ++(\barlength cm, \barheight cm);% node[pos=0.5, align=center, color=black, font=\tiny] {\texttt{LOOP\_R12\_CMP}};
			
			\pgfmathsetmacro{\barthreex}{\bartwox+\barlength}
			\pgfmathsetmacro{\barthreey}{\linetwoy-0.5*\barheight}
			
			\pgfmathsetmacro{\arrowoney}{\baroney+\barheight}
			\draw[<-] (\baronex, \arrowoney) -- ++(0, 0.2);
			
			\draw[->, anchor=east] (\bartwox, \baroney) -- ++(0, -\interbararrowheight cm);% node[pos=0.5, anchor=west] {\large\color{kityellow}\Lightning};
			
			\draw[->, anchor=east] (\barthreex, \bartwoy) -- ++(0, -0.6 cm);% node[pos=0.5, anchor=west] {\large\color{kityellow}\Lightning};
			
			\draw[decorate, decoration={brace, amplitude=3pt, mirror}] (\bartwox, \measurey) -- ++(0.6 cm,0) node[pos=0.5, anchor=north, yshift=-0.1cm, align=center] {measure \\ instructions};
			\end{tikzpicture}
			\caption{Level~2}
		\end{subfigure}
	\end{figure}
\end{frame}

\begin{frame}[t]{Findings}
	Coarse behavior (Intel Core i9-7940X):
	\begin{enumerate}
		\item Throttle out-of-order engine after first AVX instruction
		\item Switch to frequency level 1, needs $\approx$ \SI{25}{\micro\second}
		\item Switch to frequency level 2 (only heavy AVX-512 instructions), $\approx$ \SI{27}{\micro\second}
		\item Return to baseline frequency \SI[quotient-mode=fraction]{2/3}{\milli\second} after last AVX instruction
	\end{enumerate}
	\pause
	In contrast, Intel claims
	\begin{itemize}
		\item downclocking takes \enquote{up to \SI{500}{\micro\second}}
		\item upclocking after \SI{2}{\milli\second}
	\end{itemize}
\end{frame}

\begin{frame}[t]{Interesting Outlier}
	\begin{figure}
		\centering
		\begin{tikzpicture}[trim axis left, font=\footnotesize]
		\sffamily
		\begin{axis}[
		xlabel={Runs ($n=1000$)},
		ylabel={Upclocking Delay (\si{\milli\second})},
		scale only axis,
		width=9cm,
		height=3cm,
		axis lines=left,
		ymin=0.45,
		ymax=1.35,
		xtick=\empty,
		xmin=-10,
		xmax=1010,
		axis y discontinuity=parallel
		]
		\addplot[only marks, color=kitblue] table {../thesis/plots/avx_dp_fma_512_unrolled_l1_1cpus_upclock_time.csv};
		\end{axis}
		\begin{axis}[
		ylabel={\si{\micro\second}},
		scale only axis,
		width=9cm,
		height=3cm,
		hide axis,
		ymin=0.45,
		ymax=1.35,
		xtick=\empty,
		xmin=0,
		xmax=1000
		]
		\addplot[color=kitgreen, domain=0:1000, thick] {0.674378709677419} node[pos=0.03, below=0.3cm, anchor=west] {median = \SI{0.674}{\milli\second}};
		\end{axis}
		\end{tikzpicture}
	\end{figure}
	\begin{itemize}
		\item Heavy AVX-512 instructions \emph{sometimes} need up to \SI[quotient-mode=fraction]{4/3}{\milli\second} to return from level 1
		\item We do not have an explanation for this
	\end{itemize}
\end{frame}

\begin{frame}[t]{256 bit vs. 512 bit}
	\begin{figure}
		\centering
		\begin{tikzpicture}[trim axis left, font=\footnotesize]
		\sffamily
		\begin{axis}[
		xlabel={Runs ($n=1000$)},
		ylabel={Instructions},
		scale only axis,
		width=9cm,
		height=3cm,
		axis lines=left,
		ymin=0,
		ymax=32000,
		xtick=\empty,
		xmin=-10,
		xmax=1010
		]
		\addplot[only marks, color=kitblue, mark options={scale=0.5}] table {../thesis/plots/avx_dp_fma_256_unrolled_l1_1cpus_non_avx_time_avx_instructions.csv};
		\end{axis}
		\end{tikzpicture}
	\end{figure}
	\begin{itemize}
		\item 512-bit instructions always trigger frequency reductions after first instruction
		\item 256-bit ones exhibit high variance in the amount of required operations (shown above)
	\end{itemize}
\end{frame}